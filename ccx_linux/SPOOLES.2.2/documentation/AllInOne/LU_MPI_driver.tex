\vfill \eject
\section{{\tt allInOne.c} -- A Serial $LU$ Driver Program}
\label{section:LU-MPI-driver}

\begin{verbatim}
/*  allInOneMPI.c  */

#include "../spoolesMPI.h"
#include "../../timings.h"

/*--------------------------------------------------------------------*/
int
main ( int argc, char *argv[] ) {
/*
   ------------------------------------------------------------
   all-in-one MPI program for each process

   order, factor and solve A X = Y

   ( 1) read in matrix entries and form InpMtx object for A
   ( 2) order the system using minimum degree
   ( 3) permute the front tree
   ( 4) create the owners map IV object
   ( 5) permute the matrix A and redistribute
   ( 6) compute the symbolic factorization 
   ( 7) compute the numeric factorization
   ( 8) split the factors into submatrices
   ( 9) create the submatrix map and redistribute
   (10) read in right hand side entries 
        and form dense matrix DenseMtx object for Y
   (11) permute and redistribute Y
   (12) solve the linear system
   (13) gather X on processor 0

   created -- 98jun13, cca
   ------------------------------------------------------------
*/
/*--------------------------------------------------------------------*/
char            buffer[20] ;
Chv             *rootchv ;
ChvManager      *chvmanager ;
DenseMtx        *mtxX, *mtxY, *newY ;
SubMtxManager   *mtxmanager, *solvemanager ;
FrontMtx        *frontmtx ;
InpMtx          *mtxA, *newA ;
double          cutoff, droptol = 0.0, minops, tau = 100. ;
double          cpus[20] ;
double          *opcounts ;
DV              *cumopsDV ;
ETree           *frontETree ;
FILE            *inputFile, *msgFile ;
Graph           *graph ;
int             error, firsttag, ient, irow, jcol, lookahead = 0, 
                msglvl, myid, nedges, nent, neqns, nmycol, nproc, nrhs,
                nrow, pivotingflag, root, seed, symmetryflag, type ;
int             stats[20] ;
int             *rowind ;
IV              *oldToNewIV, *ownedColumnsIV, *ownersIV, 
                *newToOldIV, *vtxmapIV ;
IVL             *adjIVL, *symbfacIVL ;
SolveMap        *solvemap ;
/*--------------------------------------------------------------------*/
/*
   ---------------------------------------------------------------
   find out the identity of this process and the number of process
   ---------------------------------------------------------------
*/
MPI_Init(&argc, &argv) ;
MPI_Comm_rank(MPI_COMM_WORLD, &myid) ;
MPI_Comm_size(MPI_COMM_WORLD, &nproc) ;
/*--------------------------------------------------------------------*/
/*
   --------------------
   get input parameters
   --------------------
*/
if ( argc != 7 ) {
   fprintf(stdout, 
      "\n usage: %s msglvl msgFile type symmetryflag pivotingflag seed"
      "\n    msglvl -- message level"
      "\n    msgFile -- message file"
      "\n    type    -- type of entries"
      "\n      1 (SPOOLES_REAL)    -- real entries"
      "\n      2 (SPOOLES_COMPLEX) -- complex entries"
      "\n    symmetryflag -- type of matrix"
      "\n      0 (SPOOLES_SYMMETRIC)    -- symmetric entries"
      "\n      1 (SPOOLES_HERMITIAN)    -- Hermitian entries"
      "\n      2 (SPOOLES_NONSYMMETRIC) -- nonsymmetric entries"
      "\n    pivotingflag -- type of pivoting"
      "\n      0 (SPOOLES_NO_PIVOTING) -- no pivoting used"
      "\n      1 (SPOOLES_PIVOTING)    -- pivoting used"
      "\n    seed -- random number seed"
      "\n    "
      "\n   note: matrix entries are read in from matrix.k.input"
      "\n         where k is the process number"
      "\n   note: rhs entries are read in from rhs.k.input"
      "\n         where k is the process number"
      "\n", argv[0]) ;
   return(0) ;
}
msglvl = atoi(argv[1]) ;
if ( strcmp(argv[2], "stdout") == 0 ) {
   msgFile = stdout ;
} else {
   sprintf(buffer, "res.%d", myid) ;
   if ( (msgFile = fopen(buffer, "w")) == NULL ) {
      fprintf(stderr, "\n fatal error in %s"
              "\n unable to open file %s\n",
              argv[0], buffer) ;
      return(-1) ;
   }
}
type         = atoi(argv[3]) ;
symmetryflag = atoi(argv[4]) ;
pivotingflag = atoi(argv[5]) ;
seed         = atoi(argv[6]) ;
IVzero(20, stats) ;
DVzero(20, cpus) ;
/*--------------------------------------------------------------------*/
/*
   --------------------------------------------
   STEP 1: read the entries from the input file 
           and create the InpMtx object
   --------------------------------------------
*/
sprintf(buffer, "matrix.%d.input", myid) ;
inputFile = fopen(buffer, "r") ;
fscanf(inputFile, "%d %d %d", &neqns, &neqns, &nent) ;
mtxA = InpMtx_new() ;
InpMtx_init(mtxA, INPMTX_BY_ROWS, type, nent, 0) ;
if ( type == SPOOLES_REAL ) {
   double   value ;
   for ( ient = 0 ; ient < nent ; ient++ ) {
      fscanf(inputFile, "%d %d %le", &irow, &jcol, &value) ;
      InpMtx_inputRealEntry(mtxA, irow, jcol, value) ;
   }
} else if ( type == SPOOLES_COMPLEX ) {
   double   imag, real ;
   for ( ient = 0 ; ient < nent ; ient++ ) {
      fscanf(inputFile, "%d %d %le %le", &irow, &jcol, &real, &imag) ;
      InpMtx_inputComplexEntry(mtxA, irow, jcol, real, imag) ;
   }
}
fclose(inputFile) ;
InpMtx_sortAndCompress(mtxA) ;
InpMtx_changeStorageMode(mtxA, INPMTX_BY_VECTORS) ;
if ( msglvl > 2 ) {
   fprintf(msgFile, "\n\n input matrix") ;
   InpMtx_writeForHumanEye(mtxA, msgFile) ;
   fflush(msgFile) ;
}
/*--------------------------------------------------------------------*/
/*
   ----------------------------------------------------
   STEP 2: read the rhs entries from the rhs input file 
   and create the DenseMtx object for Y
   ----------------------------------------------------
*/
sprintf(buffer, "rhs.%d.input", myid) ;
inputFile = fopen(buffer, "r") ;
fscanf(inputFile, "%d %d", &nrow, &nrhs) ;
mtxY = DenseMtx_new() ;
DenseMtx_init(mtxY, type, 0, 0, nrow, nrhs, 1, nrow) ;
DenseMtx_rowIndices(mtxY, &nrow, &rowind) ;
if ( type == SPOOLES_REAL ) {
   double   value ;
   for ( irow = 0 ; irow < nrow ; irow++ ) {
      fscanf(inputFile, "%d", rowind + irow) ;
      for ( jcol = 0 ; jcol < nrhs ; jcol++ ) {
         fscanf(inputFile, "%le", &value) ;
         DenseMtx_setRealEntry(mtxY, irow, jcol, value) ;
      }
   }
} if ( type == SPOOLES_COMPLEX ) {
   double   imag, real ;
   for ( irow = 0 ; irow < nrow ; irow++ ) {
      fscanf(inputFile, "%d", rowind + irow) ;
      for ( jcol = 0 ; jcol < nrhs ; jcol++ ) {
         fscanf(inputFile, "%le %le", &real, &imag) ;
         DenseMtx_setComplexEntry(mtxY, irow, jcol, real, imag) ;
      }
   }
}
fclose(inputFile) ;
if ( msglvl > 2 ) {
   fprintf(msgFile, "\n\n rhs matrix in original ordering") ;
   DenseMtx_writeForHumanEye(mtxY, msgFile) ;
   fflush(msgFile) ;
}
/*--------------------------------------------------------------------*/
/*
   -------------------------------------------------------
   STEP 2 : find a low-fill ordering
   (1) create the Graph object
   (2) order the graph using multiple minimum degree
   (3) find out who has the best ordering w.r.t. op count,
       and broadcast that front tree object
   -------------------------------------------------------
*/
graph = Graph_new() ;
adjIVL = InpMtx_MPI_fullAdjacency(mtxA, stats, 
                                  msglvl, msgFile, MPI_COMM_WORLD) ;
nedges = IVL_tsize(adjIVL) ;
Graph_init2(graph, 0, neqns, 0, nedges, neqns, nedges, adjIVL,
            NULL, NULL) ;
if ( msglvl > 2 ) {
   fprintf(msgFile, "\n\n graph of the input matrix") ;
   Graph_writeForHumanEye(graph, msgFile) ;
   fflush(msgFile) ;
}
frontETree = orderViaMMD(graph, seed + myid, msglvl, msgFile) ;
Graph_free(graph) ;
if ( msglvl > 2 ) {
   fprintf(msgFile, "\n\n front tree from ordering") ;
   ETree_writeForHumanEye(frontETree, msgFile) ;
   fflush(msgFile) ;
}
opcounts = DVinit(nproc, 0.0) ;
opcounts[myid] = ETree_nFactorOps(frontETree, type, symmetryflag) ;
MPI_Allgather((void *) &opcounts[myid], 1, MPI_DOUBLE,
              (void *) opcounts, 1, MPI_DOUBLE, MPI_COMM_WORLD) ;
minops = DVmin(nproc, opcounts, &root) ;
DVfree(opcounts) ;
frontETree = ETree_MPI_Bcast(frontETree, root, 
                             msglvl, msgFile, MPI_COMM_WORLD) ;
if ( msglvl > 2 ) {
   fprintf(msgFile, "\n\n best front tree") ;
   ETree_writeForHumanEye(frontETree, msgFile) ;
   fflush(msgFile) ;
}
/*--------------------------------------------------------------------*/
/*
   -------------------------------------------------------
   STEP 3: get the permutations, permute the front tree,
           permute the matrix and right hand side.
   -------------------------------------------------------
*/
oldToNewIV = ETree_oldToNewVtxPerm(frontETree) ;
newToOldIV = ETree_newToOldVtxPerm(frontETree) ;
ETree_permuteVertices(frontETree, oldToNewIV) ;
InpMtx_permute(mtxA, IV_entries(oldToNewIV), IV_entries(oldToNewIV)) ;
if (  symmetryflag == SPOOLES_SYMMETRIC 
   || symmetryflag == SPOOLES_HERMITIAN ) { 
   InpMtx_mapToUpperTriangle(mtxA) ;
}
InpMtx_changeCoordType(mtxA, INPMTX_BY_CHEVRONS) ;
InpMtx_changeStorageMode(mtxA, INPMTX_BY_VECTORS) ;
DenseMtx_permuteRows(mtxY, oldToNewIV) ;
if ( msglvl > 2 ) {
   fprintf(msgFile, "\n\n rhs matrix in new ordering") ;
   DenseMtx_writeForHumanEye(mtxY, msgFile) ;
   fflush(msgFile) ;
}
/*--------------------------------------------------------------------*/
/*
   -------------------------------------------
   STEP 4: generate the owners map IV object
           and the map from vertices to owners
   -------------------------------------------
*/
cutoff   = 1./(2*nproc) ;
cumopsDV = DV_new() ;
DV_init(cumopsDV, nproc, NULL) ;
ownersIV = ETree_ddMap(frontETree, 
                       type, symmetryflag, cumopsDV, cutoff) ;
DV_free(cumopsDV) ;
vtxmapIV = IV_new() ;
IV_init(vtxmapIV, neqns, NULL) ;
IVgather(neqns, IV_entries(vtxmapIV), 
         IV_entries(ownersIV), ETree_vtxToFront(frontETree)) ;
if ( msglvl > 2 ) {
   fprintf(msgFile, "\n\n map from fronts to owning processes") ;
   IV_writeForHumanEye(ownersIV, msgFile) ;
   fprintf(msgFile, "\n\n map from vertices to owning processes") ;
   IV_writeForHumanEye(vtxmapIV, msgFile) ;
   fflush(msgFile) ;
}
/*--------------------------------------------------------------------*/
/*
   ---------------------------------------------------
   STEP 5: redistribute the matrix and right hand side
   ---------------------------------------------------
*/
firsttag = 0 ;
newA = InpMtx_MPI_split(mtxA, vtxmapIV, stats, 
                        msglvl, msgFile, firsttag, MPI_COMM_WORLD) ;
firsttag++ ;
InpMtx_free(mtxA) ;
mtxA = newA ;
InpMtx_changeStorageMode(mtxA, INPMTX_BY_VECTORS) ;
if ( msglvl > 2 ) {
   fprintf(msgFile, "\n\n split InpMtx") ;
   InpMtx_writeForHumanEye(mtxA, msgFile) ;
   fflush(msgFile) ;
}
newY = DenseMtx_MPI_splitByRows(mtxY, vtxmapIV, stats, msglvl, 
                                msgFile, firsttag, MPI_COMM_WORLD) ;
DenseMtx_free(mtxY) ;
mtxY = newY ;
firsttag += nproc ;
if ( msglvl > 2 ) {
   fprintf(msgFile, "\n\n split DenseMtx Y") ;
   DenseMtx_writeForHumanEye(mtxY, msgFile) ;
   fflush(msgFile) ;
}
/*--------------------------------------------------------------------*/
/*
   ------------------------------------------
   STEP 6: compute the symbolic factorization
   ------------------------------------------
*/
symbfacIVL = SymbFac_MPI_initFromInpMtx(frontETree, ownersIV, mtxA,
                     stats, msglvl, msgFile, firsttag, MPI_COMM_WORLD) ;
firsttag += frontETree->nfront ;
if ( msglvl > 2 ) {
   fprintf(msgFile, "\n\n local symbolic factorization") ;
   IVL_writeForHumanEye(symbfacIVL, msgFile) ;
   fflush(msgFile) ;
}
/*--------------------------------------------------------------------*/
/*
   -----------------------------------
   STEP 7: initialize the front matrix
   -----------------------------------
*/
mtxmanager = SubMtxManager_new() ;
SubMtxManager_init(mtxmanager, NO_LOCK, 0) ;
frontmtx = FrontMtx_new() ;
FrontMtx_init(frontmtx, frontETree, symbfacIVL, type, symmetryflag,
              FRONTMTX_DENSE_FRONTS, pivotingflag, NO_LOCK, myid,
              ownersIV, mtxmanager, msglvl, msgFile) ;
/*--------------------------------------------------------------------*/
/*
   ---------------------------------
   STEP 8: compute the factorization
   ---------------------------------
*/
chvmanager = ChvManager_new() ;
ChvManager_init(chvmanager, NO_LOCK, 0) ;
rootchv = FrontMtx_MPI_factorInpMtx(frontmtx, mtxA, tau, droptol,
                     chvmanager, ownersIV, lookahead, &error, cpus, 
                     stats, msglvl, msgFile, firsttag, MPI_COMM_WORLD) ;
ChvManager_free(chvmanager) ;
firsttag += 3*frontETree->nfront + 2 ;
if ( msglvl > 2 ) {
   fprintf(msgFile, "\n\n numeric factorization") ;
   FrontMtx_writeForHumanEye(frontmtx, msgFile) ;
   fflush(msgFile) ;
}
if ( error >= 0 ) {
   fprintf(stderr, 
          "\n proc %d : factorization error at front %d", myid, error) ;
   MPI_Finalize() ;
   exit(-1) ;
}
/*--------------------------------------------------------------------*/
/*
   ------------------------------------------------
   STEP 9: post-process the factorization and split 
   the factor matrices into submatrices 
   ------------------------------------------------
*/
FrontMtx_MPI_postProcess(frontmtx, ownersIV, stats, msglvl,
                         msgFile, firsttag, MPI_COMM_WORLD) ;
firsttag += 5*nproc ;
if ( msglvl > 2 ) {
   fprintf(msgFile, "\n\n numeric factorization after post-processing");
   FrontMtx_writeForHumanEye(frontmtx, msgFile) ;
   fflush(msgFile) ;
}
/*--------------------------------------------------------------------*/
/*
   -----------------------------------
   STEP 10: create the solve map object
   -----------------------------------
*/
solvemap = SolveMap_new() ;
SolveMap_ddMap(solvemap, symmetryflag, 
               FrontMtx_upperBlockIVL(frontmtx),
               FrontMtx_lowerBlockIVL(frontmtx),
               nproc, ownersIV, FrontMtx_frontTree(frontmtx), 
               seed, msglvl, msgFile);
if ( msglvl > 3 ) {
   SolveMap_writeForHumanEye(solvemap, msgFile) ;
   fflush(msgFile) ;
}
/*--------------------------------------------------------------------*/
/*
   ----------------------------------------------------
   STEP 11: redistribute the submatrices of the factors
   ----------------------------------------------------
*/
FrontMtx_MPI_split(frontmtx, solvemap, 
                   stats, msglvl, msgFile, firsttag, MPI_COMM_WORLD) ;
if ( msglvl > 2 ) {
   fprintf(msgFile, "\n\n numeric factorization after split") ;
   FrontMtx_writeForHumanEye(frontmtx, msgFile) ;
   fflush(msgFile) ;
}
/*--------------------------------------------------------------------*/
/*
   ------------------------------------------------
   STEP 13: permute and redistribute Y if necessary
   ------------------------------------------------
*/
if ( FRONTMTX_IS_PIVOTING(frontmtx) ) {
   IV   *rowmapIV ;
/*
   ----------------------------------------------------------
   pivoting has taken place, redistribute the right hand side
   to match the final rows and columns in the fronts
   ----------------------------------------------------------
*/
   rowmapIV = FrontMtx_MPI_rowmapIV(frontmtx, ownersIV, msglvl,
                                    msgFile, MPI_COMM_WORLD) ;
   newY = DenseMtx_MPI_splitByRows(mtxY, rowmapIV, stats, msglvl, 
                                   msgFile, firsttag, MPI_COMM_WORLD) ;
   DenseMtx_free(mtxY) ;
   mtxY = newY ;
   IV_free(rowmapIV) ;
}
if ( msglvl > 2 ) {
   fprintf(msgFile, "\n\n rhs matrix after split") ;
   DenseMtx_writeForHumanEye(mtxY, msgFile) ;
   fflush(msgFile) ;
}
/*--------------------------------------------------------------------*/
/*
   ------------------------------------------
   STEP 14: create a solution DenseMtx object
   ------------------------------------------
*/
ownedColumnsIV = FrontMtx_ownedColumnsIV(frontmtx, myid, ownersIV,
                                         msglvl, msgFile) ;
nmycol = IV_size(ownedColumnsIV) ;
mtxX = DenseMtx_new() ;
if ( nmycol > 0 ) {
   DenseMtx_init(mtxX, type, 0, 0, nmycol, nrhs, 1, nmycol) ;
   DenseMtx_rowIndices(mtxX, &nrow, &rowind) ;
   IVcopy(nmycol, rowind, IV_entries(ownedColumnsIV)) ;
}
/*--------------------------------------------------------------------*/
/*
   --------------------------------
   STEP 15: solve the linear system
   --------------------------------
*/
solvemanager = SubMtxManager_new() ;
SubMtxManager_init(solvemanager, NO_LOCK, 0) ;
FrontMtx_MPI_solve(frontmtx, mtxX, mtxY, solvemanager, solvemap, cpus, 
                   stats, msglvl, msgFile, firsttag, MPI_COMM_WORLD) ;
SubMtxManager_free(solvemanager) ;
if ( msglvl > 2 ) {
   fprintf(msgFile, "\n solution in new ordering") ;
   DenseMtx_writeForHumanEye(mtxX, msgFile) ;
}
/*--------------------------------------------------------------------*/
/*
   --------------------------------------------------------
   STEP 15: permute the solution into the original ordering
            and assemble the solution onto processor zero
   --------------------------------------------------------
*/
DenseMtx_permuteRows(mtxX, newToOldIV) ;
if ( msglvl > 2 ) {
   fprintf(msgFile, "\n\n solution in old ordering") ;
   DenseMtx_writeForHumanEye(mtxX, msgFile) ;
   fflush(msgFile) ;
}
IV_fill(vtxmapIV, 0) ;
firsttag++ ;
mtxX = DenseMtx_MPI_splitByRows(mtxX, vtxmapIV, stats, msglvl, msgFile,
                                firsttag, MPI_COMM_WORLD) ;
if ( myid == 0 && msglvl > 0 ) {
   fprintf(msgFile, "\n\n complete solution in old ordering") ;
   DenseMtx_writeForHumanEye(mtxX, msgFile) ;
   fflush(msgFile) ;
}
/*--------------------------------------------------------------------*/
MPI_Finalize() ;

return(1) ; }
/*--------------------------------------------------------------------*/
\end{verbatim}
